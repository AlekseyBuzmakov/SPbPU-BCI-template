Приведём пример табличного представления данных с записью продолжения на следующей странице, см. таблицу \ref{tab:long}.

\begingroup
\centering
\smallA %выставляем шрифт в 13bp
\begin{longtable}[c]{|l|l|l|l|l|l|}
	\caption{Пример задания данных из \cite{Peskov2004} (с повтором для переноса таблицы на новую страницу)}%
	\label{tab:long}% label всегда желательно идти после caption
	\\
	\hline
	$G$&$m_1$&$m_2$&$m_3$&$m_4$&$K$\\ \hline
	\endfirsthead%
	\captionsetup{format=tablenocaption,labelformat=continued}% должен стоять до самого caption
	\caption[]{}\\
	\hline
	$G$&$m_1$&$m_2$&$m_3$&$m_4$&$K$\\ \hline
	\endhead
	\hline
	\endfoot
	\hline
	\endlastfoot
	$g_1$&0&1&1&0&1\\
	$g_2$&1&2&0&1&1\\
	$g_3$&0&1&0&1&1\\
	$g_4$&1&2&1&0&2\\
	$g_5$&1&1&0&1&2\\
	$g_6$&1&1&1&2&2\\
	\hline
	$g_1$&0&1&1&0&1\\
	$g_2$&1&2&0&1&1\\
	$g_3$&0&1&0&1&1\\
	$g_4$&1&2&1&0&2\\
	$g_5$&1&1&0&1&2\\
	$g_6$&1&1&1&2&2\\
	\hline
	$g_1$&0&1&1&0&1\\
	$g_2$&1&2&0&1&1\\
	$g_3$&0&1&0&1&1\\
	$g_4$&1&2&1&0&2\\
	$g_5$&1&1&0&1&2\\
	$g_6$&1&1&1&2&2\\
		\hline
	$g_1$&0&1&1&0&1\\
	$g_2$&1&2&0&1&1\\
	$g_3$&0&1&0&1&1\\
	$g_4$&1&2&1&0&2\\
	$g_5$&1&1&0&1&2\\
	$g_6$&1&1&1&2&2\\
	\hline
	$g_1$&0&1&1&0&1\\
	$g_2$&1&2&0&1&1\\
	$g_3$&0&1&0&1&1\\
	$g_4$&1&2&1&0&2\\
	$g_5$&1&1&0&1&2\\
	$g_6$&1&1&1&2&2\\
\end{longtable}
\normalsize% возвращаем шрифт к нормальному
\endgroup
\normalsize% возвращаем шрифт к нормальному