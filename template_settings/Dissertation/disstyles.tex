%%% Изображения %%%
\graphicspath{{images/}{Dissertation/images/}}         % Пути к изображениям

%%% Макет страницы %%%
% Выставляем значения полей (ГОСТ 7.0.11-2011, 5.3.7)
\makeatletter
\geometry{a4paper,top=2cm,bottom=2cm,left=2cm,right=2cm,
	headsep=0cm, %отступ от колонтитула от живописного поля
%	head=2cm, %верхняя граница колонтитула
	headheight=0cm,
%	nofoot,
includefoot,
	nomarginpar
%	,showframe
} 
%\setlength{\topskip}{0pt}   %размер дополнительного верхнего поля
\makeatother

%%% Интервалы %%%
%% По ГОСТ Р 7.0.11-2011, пункту 5.3.6 требуется полуторный интервал
%% Реализация средствами класса (на основе setspace) ближе к типографской классике.
%% И правит сразу и в таблицах (если со звёздочкой) 
%\DoubleSpacing*     % Двойной интервал
\OnehalfSpacing*    % Полуторный интервал
%\setSpacing{1.42}   % Полуторный интервал, подобный Ворду (возможно, стоит включать вместе с предыдущей строкой)

%%% Выравнивание и переносы %%%
%% http://tex.stackexchange.com/questions/241343/what-is-the-meaning-of-fussy-sloppy-emergencystretch-tolerance-hbadness
%% http://www.latex-community.org/forum/viewtopic.php?p=70342#p70342
\tolerance 1414
\hbadness 1414
\emergencystretch 1.5em % В случае проблем регулировать в первую очередь
\hfuzz 0.3pt
\vfuzz \hfuzz
%\raggedbottom
%\sloppy                 % Избавляемся от переполнений
\clubpenalty=10000      % Запрещаем разрыв страницы после первой строки абзаца
\widowpenalty=10000     % Запрещаем разрыв страницы после последней строки абзаца

%%% Блок управления параметрами для выравнивания заголовков в тексте %%%
\newlength{\otstuplen}
\setlength{\otstuplen}{\theotstup\parindent}
\ifnumequal{\value{headingalign}}{0}{% выравнивание заголовков в тексте
    \newcommand{\hdngalign}{\centering}                % по центру
    \newcommand{\hdngaligni}{}% по центру
    \setlength{\otstuplen}{0pt}
}{%
    \newcommand{\hdngalign}{}                 % по левому краю
    \newcommand{\hdngaligni}{\hspace{\otstuplen}}      % по левому краю
} % В обоих случаях вроде бы без переноса, как и надо (ГОСТ Р 7.0.11-2011, 5.3.5)







%%% Оглавление %%%

\renewcommand{\cftchapterdotsep}{\cftdotsep}                % отбивка точками до номера страницы начала главы/раздела

%% Переносить слова в заголовке не допускается (ГОСТ Р 7.0.11-2011, 5.3.5). Заголовки в оглавлении должны точно повторять заголовки в тексте (ГОСТ Р 7.0.11-2011, 5.2.3). Прямого указания на запрет переносов в оглавлении нет, но по той же логике невнесения искажений в смысл, лучше в оглавлении не переносить:
%\setrmarg{2.55em plus1fil}                             %To have the (sectional) titles in the ToC, etc., typeset ragged right with no hyphenation


%% снятие жирности %%
%\cftKleader defines the leader between the title and the page number; it can be
%changed by \renewcommand. The spacing between any dots in the leader is controlled
%by \cftKdotsep
\makeatletter
\renewcommand{\cftchapterpagefont}{\normalfont}        % нежирные номера страниц у глав в оглавлении
\renewcommand{\cftchapterleader}{\cftdotfill{\cftchapterdotsep}}% нежирные точки до номеров страниц у глав в оглавлении
\renewcommand{\cftchapterfont}{}                       % нежирные названия глав в оглавлении
\renewcommand{\cftchapterpagefont}{}                       % нежирные названия номеров глав в оглавлении

%сейчас реализован <<ручной>> ввод в Содержание части, поэтому следующий код <<в разработке>>.
%TO-DO: необходимо подкорректировать код для передачи {part} in ToC так, чтобы 
% 1) отображалась римская цифра I.
% 2) осуществлялся отсутп от этой цифры на примере chapter-section etc.

%\renewcommand{\cftpartdotsep}{\cftdotsep}                % отбивка точками до номера страницы начала части
%\renewcommand{\cftpartpagefont}{\normalfont}        % нежирные номера страниц у частей в оглавлении
%\renewcommand{\cftpartleader}{\cftdotfill{\cftpartdotsep}}% нежирные точки до номеров страниц у частей в оглавлении
%\renewcommand{\cftpartfont}{}                       % нежирные названия частей в оглавлении
%\renewcommand{\cftpartpagefont}{}                       % нежирные названия номеров частей в оглавлении
%\renewcommand{\cftpartname}{Part\space}
%\renewcommand\cftpartaftersnum{.\space}       % добавляет точку с пробелом после номера раздела в оглавлении

%%ниже вариант форматирвования от daleif, подключив который видно, что проблемы с переносом SNUM.
%%https://tex.stackexchange.com/a/210676/44348
%\renewcommand{\partnumberline}[1]{Teil #1:~}
%\renewcommand*{\l@part}[2]{%
%	\ifnum \c@tocdepth >-2\relax
%	\cftpartbreak
%	\begingroup
%	{
%		\setlength{\memRTLleftskip}{0pt}
%		\setlength{\memRTLrightskip}{0pt}
%		\interlinepenalty\@M
%		\centering
%		\cftpartfont #1
%		\par
%	}
%	\nobreak
%	\global\@nobreaktrue
%	\everypar{\global\@nobreakfalse\everypar{}}%
%	\endgroup
%	\fi}
%\renewcommand{\cftpartfont}{\large\normalfont}
\makeatother


%% Форматирование SPbPU %%
%https://tex.stackexchange.com/questions/394227/memoir-toc-indent-the-second-line-by-numberspace-width-in-the-previous-line-or
		\renewcommand{\mempreaddchaptertotochook}{%
			\addtocontents{toc}{%
				\protect\settowidth{\protect\cftchapternumwidth}{\cftchapterfont \thechapter.\space}%
				\protect\setlength{\protect\cftsectionindent}{\protect\cftchapternumwidth}%
			}%
			\addtocontents{tec}{%
				\protect\settowidth{\protect\cftchapternumwidth}{\cftchapterfont \thechapter.\space}%
				\protect\setlength{\protect\cftsectionindent}{\protect\cftchapternumwidth}%
			}%
			\addtocontents{tuc}{%
			\protect\settowidth{\protect\cftchapternumwidth}{\cftchapterfont \thechapter.\space}%
			\protect\setlength{\protect\cftsectionindent}{\protect\cftchapternumwidth}%
		}%
		}
%		\renewcommand{\mempreaddparttotochook}{%
%		\addtocontents{toc}{%
%		\protect\settowidth{\protect\cftpartnumwidth}{\cftpartfont \thepart.\space}%
%		\protect\settowidth{\protect\cftchapternumwidth}{\cftchapterfont \thechapter.\space}%
%		\protect\setlength{\protect\cftsectionindent}{\protect\cftchapternumwidth}%
%				}%
%		}		
		\usepackage{xpatch}
		\makeatletter
		%% for standard ToC called <<toc>>
		\xpatchcmd{\@m@mschapter}% <cmd>
		{\@schapter}% <search>
		{\addtocontents{toc}{\protect\settowidth{\protect\cftsectionindent}{\protect\cftchapterfont \thechapter.\space}}%
			\@schapter}% <replace>
		{}{}% <success><failure>
		\patchcmd{\M@sect}% <cmd>
		{\addcontentsline}% <search>
		{\addtocontents{toc}{%
				\protect\settowidth{\csname cft#1numwidth\endcsname}{\csname cft#1font\endcsname\csname the#1\endcsname.\space}%
				\protect\setlength{\csname cftsub#1indent\endcsname}{\protect\dimexpr\csname cft#1numwidth\endcsname+\csname cft#1indent\endcsname}%
			}%
			\addcontentsline}% <replace>
		{}{}% <success><failure>
		%% for ToC called <<tec>>
		\xpatchcmd{\@m@mschapter}% <cmd>
		{\@schapter}% <search>
		{\addtocontents{tec}{\protect\settowidth{\protect\cftsectionindent}{\protect\cftchapterfont \thechapter.\space}}%
			\@schapter}% <replace>
		{}{}% <success><failure>
		\patchcmd{\M@sect}% <cmd>
		{\addcontentsline}% <search>
		{\addtocontents{tec}{%
				\protect\settowidth{\csname cft#1numwidth\endcsname}{\csname cft#1font\endcsname\csname the#1\endcsname.\space}%
				\protect\setlength{\csname cftsub#1indent\endcsname}{\protect\dimexpr\csname cft#1numwidth\endcsname+\csname cft#1indent\endcsname}%
			}%
			\addcontentsline}% <replace>
		{}{}% <success><failure>
		%% for ToC called <<tuc>>
		\xpatchcmd{\@m@mschapter}% <cmd>
		{\@schapter}% <search>
		{\addtocontents{tuc}{\protect\settowidth{\protect\cftsectionindent}{\protect\cftchapterfont \thechapter.\space}}%
			\@schapter}% <replace>
		{}{}% <success><failure>
		\patchcmd{\M@sect}% <cmd>
		{\addcontentsline}% <search>
		{\addtocontents{tuc}{%
				\protect\settowidth{\csname cft#1numwidth\endcsname}{\csname cft#1font\endcsname\csname the#1\endcsname.\space}%
				\protect\setlength{\csname cftsub#1indent\endcsname}{\protect\dimexpr\csname cft#1numwidth\endcsname+\csname cft#1indent\endcsname}%
			}%
			\addcontentsline}% <replace>
		{}{}% <success><failure>
		\makeatother

		%% delete boxes
		\renewcommand\numberlinebox[2]{#2} % for sections
		\renewcommand\chapternumberlinebox[2]{#2} % for chapters 
%		\renewcommand\partnumberlinebox[2]{#2} % for parts 

		%% работа с расстояниями между точками, переносами слов
		\makeatletter
		\renewcommand{\cftdotsep}{0.1}
		%\renewcommand{\@dotset}{0.1} %old macro DOES NOT WORK
		\setpnumwidth{0.7em} 
		%set 1.0 em to get small equal indent <-- in this case there will not be the exceeding the left marlin 2cm.
		%\renewcommand{\@pnumwidth}{0em} %old macro
		\setrmarg{1.5em plus1fil} 
		%set 2.55 em or similar to restrict hyphenation
		%plus1fil makes the distance between the words smaller!
		%it helps to make the equal indent
		%\renewcommand{\@tocrmarg}{1.5em plus1fil} %old macro
		\makeatother
		
		%% отступы %%
		%\renewcommand{\cftchapterbreak}{}        % set a page break before rather than after the entry
		%\renewcommand{\cftparskip}{\parskip} %parskip in ToC etc. %seems to be the same as default or do nothing
		\setlength{\cftbeforechapterskip}{0pt plus 0pt} %delete skip after chapter block (last section) %%%<-SPbPU pure
		\setlength{\cftbeforepartskip}{0pt plus 0pt} %delete skip after chapter block (last section) %%%<-SPbPU pure
		
		
		
%% Продолжение редактирования оглавления настройками CandDoctDiss %%		


\ifnumgreater{\value{headingdelim}}{0}{%
	%<- SPbPU точка после номера страницы
    \renewcommand\cftchapteraftersnum{.\space}       % добавляет точку с пробелом после номера раздела в оглавлении
%\renewcommand\cftchapteraftersnumb{\enspace\textperiodcentered\enspace} %\enspace - настоящий пробел, \space не работает
%\renewcommand\chapternumberlinebox[2]{#2}
}{}
\ifnumgreater{\value{headingdelim}}{1}{%%%<-SPbPU 
%	   	\renewcommand\cftsectionpresnum{..}       % добавляет smth перед number %выравнивает в box
    % точка после номера страницы
    \renewcommand\cftsectionaftersnum{.\space}       % добавляет точку с пробелом после номера подраздела в оглавлении
% last is \hfil !
%   	\renewcommand\cftsectionaftersnumb{...}       % добавляет точки перед названием %можно удалить пробел
       \renewcommand\cftsubsectionaftersnum{.\space}    % добавляет точку с пробелом после номера подподраздела в оглавлении
    \renewcommand\cftsubsubsectionaftersnum{.\space} % добавляет точку с пробелом после номера подподподраздела в оглавлении
    \AtBeginDocument{% без этого polyglossia сама всё переопределяет
        \setsecnumformat{\csname the#1\endcsname.\space}
        %\setsecnumformat{\csname the#1\endcsname:\quad}
    }
}{%
    \AtBeginDocument{% без этого polyglossia сама всё переопределяет
        \setsecnumformat{\csname the#1\endcsname\quad} %
    }
}

\renewcommand*{\cftappendixname}{\appendixname\space} % Слово Приложение в оглавлении


%%% Различные варианты форматирования SPbPU %%%

%% форматирование отсупов до номеров страниц стр. 151 мемуара !!!
%\renewcommand*{\cftsectionnumwidth}{} %выставление абсолютного значения
%\addtolength{\cftsectionnumwidth}{5em} %не работает

%убираем фиксированные размеры of the box %%%<-SPbPU pure
%\AtBeginDocument{%
%\renewcommand\numberlinebox[2]{#2} % for sections and for chapters in Russian ?! %%%<-SPbPU pure
%\renewcommand\chapternumberlinebox[2]{#2} % for chapters in English ? %% report about the bug to the package mainteners
%\newcommand\chapternumberlinebox[2]{%
%	\hb@xt@#1{#2\hfil}}

%\newcommand\chapternumberlinebox[2]{%
%	#1{\hfil#2}}

%\numberlinebox{hlengthi}{hcodei} %выставление абсолютного значения
%\chapternumberlinebox{hlengthi}{hcodei} %выставление абсолютного значения
%}

%убираем растояния до \cftsectionpresnum в размере одного абзацного отступа %%%<-SPbPU pure
%%\cftsetindents{hkindi}{hindenti}{hnumwidthi}


%https://tex.stackexchange.com/questions/306851/multiline-unnumbered-chapter-in-table-of-contents
%https://tex.stackexchange.com/questions/40022/customized-table-of-contents-same-indentation-for-every-line-of-multi-line-titl
%\parindent % standart padding
% это здорово экономит место, но нужно тогда синхронизировать стиль обычных отступов в перечислениях
% недостаток - не видно выравнивания по первому слову в названии предыдущего раздела
%\AtBeginDocument{
%	\cftsetindents{chapter}{0pt}{% первая строка
%		0pt} % последующие строки от первой
%	\cftsetindents{section}{%
%		\parindent
%%3.5ex plus 1ex minus .2ex
%	}{%
%		0em
%%2.3ex plus .2ex
%}
%	\cftsetindents{subsection}{%
%		\cftsectionindent+\parindent}{% т.е. 2 абзацных отступа
%		0em}
%	\cftsetindents{subsubsection}{%
%		\cftsubsectionindent+\parindent}{% т.е. 3 абзацных отступа
%		0em}
%}




%%% Колонтитулы %%%
% Порядковый номер страницы печатают на середине верхнего поля страницы (ГОСТ Р 7.0.11-2011, 5.3.8)
%сделаем справа внизу
%\makeatletter
\makeevenhead{plain}{}{}{}
\makeoddhead{plain}{}{}{}
\makeevenfoot{plain}{}{\thepage}{}
\makeoddfoot{plain}{}{\thepage}{}
\pagestyle{plain}

%%% добавить Стр. над номерами страниц в оглавлении
%%% http://tex.stackexchange.com/a/306950
\newif\ifendTOC
%
\newcommand*{\tocheader}{
%\ifnumequal{\value{pgnum}}{1}{%
%    \ifendTOC\else\hbox to \linewidth%
%      {\noindent{}~\hfill{Pages}}\par%
%      \ifnumless{\value{page}}{3}{}{%
%        \vspace{0.5\onelineskip}
%      }
%      \afterpage{\tocheader}
%    \fi%
%}{}%
}%


%epigraph with DOI
%\usepackage{quotchap}




%%% SPbPU %%% Оформление заголовков глав, разделов, подразделов %%%

\makechapterstyle{SPbPUstyle}{%
	\chapterstyle{default}
	\setlength{\beforechapskip}{0pt}
	\setlength{\midchapskip}{0pt}
	\setlength{\afterchapskip}{\theintvl\curtextsize}
	\renewcommand*{\chapnamefont}{\normalfont\bfseries\MakeTextUppercase} %не используется слово <<Глава>>
	\renewcommand*{\chapnumfont}{\normalfont\bfseries\MakeTextUppercase}
%	\renewcommand*{\chaptitlefont}{\normalfont\bfseries\MakeTextUppercase} %не работает \MakeTextUppercase
	\renewcommand\printchaptertitle{\normalfont\bfseries\MakeTextUppercase}% аналог \chaptitlefont
	\renewcommand*{\chapterheadstart}{}
	\ifnumgreater{\value{headingdelim}}{0}{%
		\renewcommand*{\afterchapternum}{.\space}   % добавляет точку с пробелом после номера раздела
	}{%
		\renewcommand*{\afterchapternum}{.\quad}     % добавляет точку и \space (\quad) после номера раздела
	} % настройки добавление в СОДЕРЖАНИЕ (по умолчанию название раздела переходит само)
	\renewcommand*{\printchapternum}{\hdngaligni\hdngalign\chapnumfont \thechapter}
	\renewcommand*{\printchaptername}{}
	\renewcommand*{\printchapternonum}{\hdngaligni\hdngalign}
	}
\newcommand{\chapterLight}{\normalfont\MakeTextUppercase\normalsize} %не менять последовательность команд!
\renewcommand*{\printtoctitle}[1]{\normalfont\MakeTextUppercase #1} %слово <<Content>> в стилю chaperLight, по факту убираем \bfseries
%\chapterLight не действует в этой команде
\makeatletter


\makechapterstyle{SPbPUstylechapname}{% для <<будет вписано слово Глава перед каждым номером раздела в оглавлении>>
	\chapterstyle{SPbPUstyle}
	\renewcommand*{\printchapternum}{\chapnumfont \thechapter}
	\renewcommand*{\printchaptername}{\hdngaligni\hdngalign\chapnamefont \@chapapp} %

}
\makeatother

\chapterstyle{SPbPUstyle}

%% удалить перенос на новую строку перед командой \chapter
\newcommand{\delnewpagebeforech}{
	%\begingroup
	\renewcommand{\cleardoublepage}{}
	\renewcommand{\clearpage}{}
	\vspace{\theintvl\curtextsize}
	%\endgroup %after chapter in case of inline using
}



%% Оформление шрифтов и отсупов подразделов, подподразделов и подподподразделов

\makeatletter
\setsecheadstyle{\normalfont\bfseries\hdngalign}
\setsecindent{\otstuplen} %отступ от левого края живописного поля
\setbeforesecskip{2\curtextsize} %базовы настройки с плюс/минус точностью, что позволяет более гибко располагать рисунки и изображения на странице
\setaftersecskip{2\curtextsize}


\setsubsecheadstyle{\normalfont\bfseries\itshape\hdngalign}
\setsubsecindent{\otstuplen}
\setbeforesubsecskip{1\curtextsize}
\setaftersubsecskip{1\curtextsize}

\setsubsubsecheadstyle{\normalfont\itshape\hdngalign}
\setsubsubsecindent{\otstuplen}
\setbeforesubsubsecskip{1\curtextsize}
\setaftersubsubsecskip{1\curtextsize}
\makeatother

%попытки форматирования \part можно продолжить
%сейчас реализован более простой вариант
\renewcommand{\partnamefont}{\LARGE\MakeTextUppercase}
\renewcommand{\partnumfont}{\LARGE\MakeTextUppercase}
\renewcommand*{\parttitlefont}{\LARGE\MakeTextUppercase}

%[section], чтобы заставить все floats быть до расположиться до окончания подраздела
%\FloatBarrier локальное ограничение, чтобы 
% расставить повсеместно по разделам, то всего лишь подключить [section];
% разрешить до \FloatBarrier размещать foats, то добавить окцию  [above].
\usepackage[above]{placeins} 

\sethangfrom{\noindent #1} %все заголовки подразделов центрируются с учетом номера, как block 

\ifnumequal{\value{chapstyle}}{1}{%
    \chapterstyle{SPbPUstylechapname}
%    \renewcommand*{\cftchaptername}{Chapter\space} % будет вписано слово Глава перед каждым номером раздела в оглавлении
}{}% вместо Chapter \chaptername

%%% Интервалы между заголовками
%\setbeforesecskip{\theintvl\curtextsize}% Заголовки отделяют от текста сверху и снизу тремя интервалами (ГОСТ Р 7.0.11-2011, 5.3.5).
%\setaftersecskip{\theintvl\curtextsize}
%\setbeforesubsecskip{\theintvl\curtextsize}
%\setaftersubsecskip{\theintvl\curtextsize}
%\setbeforesubsubsecskip{\theintvl\curtextsize}
%\setaftersubsubsecskip{\theintvl\curtextsize}


%%% Блок дополнительного управления размерами заголовков
\ifnumequal{\value{headingsize}}{1}{% Пропорциональные заголовки и базовый шрифт 14 пт
	\renewcommand{\normalfont}{\large\bfseries}
	\renewcommand*{\chapnamefont}{\Large\bfseries}
	\renewcommand*{\chapnumfont}{\Large\bfseries}
	\renewcommand*{\chaptitlefont}{\Large\bfseries}
}{}




%%% Счётчики %%%

%% DOI
\newcounter{mychapternumber} 
\newcounter{chapterDOI}

%% Упрощённые настройки шаблона диссертации: нумерация формул, таблиц, рисунков
\ifnumequal{\value{contnumeq}}{1}{%
    \counterwithout{equation}{chapter} % Убираем связанность номера формулы с номером главы/раздела
}{}
\ifnumequal{\value{contnumfig}}{1}{%
    \counterwithout{figure}{chapter}   % Убираем связанность номера рисунка с номером главы/раздела
}{}
\ifnumequal{\value{contnumtab}}{1}{%
    \counterwithout{table}{chapter}    % Убираем связанность номера таблицы с номером главы/раздела
}{}


%%http://www.linux.org.ru/forum/general/6993203#comment-6994589 (используется totcount)
\makeatletter
\def\formbytotal#1#2#3#4#5{%
    \newcount\@c
    \@c\totvalue{#1}\relax
    \newcount\@last
    \newcount\@pnul
    \@last\@c\relax
    \divide\@last 10
    \@pnul\@last\relax
    \divide\@pnul 10
    \multiply\@pnul-10
    \advance\@pnul\@last
    \multiply\@last-10
    \advance\@last\@c
    \total{#1}~#2%
    \ifnum\@pnul=1#5\else%
    \ifcase\@last#5\or#3\or#4\or#4\or#4\else#5\fi
    \fi
}
\makeatother

\AtBeginDocument{
%% регистрируем счётчики в системе totcounter
    \regtotcounter{totalcount@figure}
    \regtotcounter{totalcount@table}       % Если иным способом поставить в преамбуле то ошибка в числе таблиц
    \regtotcounter{TotPages}               % Если иным способом поставить в преамбуле то ошибка в числе страниц
}

%%% Правильная нумерация приложений %%%
%% По ГОСТ 2.105, п. 4.3.8 Приложения обозначают заглавными буквами русского алфавита,
%% начиная с А, за исключением букв Ё, З, Й, О, Ч, Ь, Ы, Ъ.
%% Здесь также переделаны все нумерации русскими буквами.
\ifxetexorluatex
    \makeatletter
    \def\russian@Alph#1{\ifcase#1\or
       А\or Б\or В\or Г\or Д\or Е\or Ж\or
       И\or К\or Л\or М\or Н\or
       П\or Р\or С\or Т\or У\or Ф\or Х\or
       Ц\or Ш\or Щ\or Э\or Ю\or Я\else\xpg@ill@value{#1}{russian@Alph}\fi}
    \def\russian@alph#1{\ifcase#1\or
       а\or б\or в\or г\or д\or е\or ж\or
       и\or к\or л\or м\or н\or
       п\or р\or с\or т\or у\or ф\or х\or
       ц\or ш\or щ\or э\or ю\or я\else\xpg@ill@value{#1}{russian@alph}\fi}
    \makeatother
\else
    \makeatletter
    \if@uni@ode
      \def\russian@Alph#1{\ifcase#1\or
        А\or Б\or В\or Г\or Д\or Е\or Ж\or
        И\or К\or Л\or М\or Н\or
        П\or Р\or С\or Т\or У\or Ф\or Х\or
        Ц\or Ш\or Щ\or Э\or Ю\or Я\else\@ctrerr\fi}
    \else
      \def\russian@Alph#1{\ifcase#1\or
        \CYRA\or\CYRB\or\CYRV\or\CYRG\or\CYRD\or\CYRE\or\CYRZH\or
        \CYRI\or\CYRK\or\CYRL\or\CYRM\or\CYRN\or
        \CYRP\or\CYRR\or\CYRS\or\CYRT\or\CYRU\or\CYRF\or\CYRH\or
        \CYRC\or\CYRSH\or\CYRSHCH\or\CYREREV\or\CYRYU\or
        \CYRYA\else\@ctrerr\fi}
    \fi
    \if@uni@ode
      \def\russian@alph#1{\ifcase#1\or
        а\or б\or в\or г\or д\or е\or ж\or
        и\or к\or л\or м\or н\or
        п\or р\or с\or т\or у\or ф\or х\or
        ц\or ш\or щ\or э\or ю\or я\else\@ctrerr\fi}
    \else
      \def\russian@alph#1{\ifcase#1\or
        \cyra\or\cyrb\or\cyrv\or\cyrg\or\cyrd\or\cyre\or\cyrzh\or
        \cyri\or\cyrk\or\cyrl\or\cyrm\or\cyrn\or
        \cyrp\or\cyrr\or\cyrs\or\cyrt\or\cyru\or\cyrf\or\cyrh\or
        \cyrc\or\cyrsh\or\cyrshch\or\cyrerev\or\cyryu\or
        \cyrya\else\@ctrerr\fi}
    \fi
    \makeatother
\fi


%%% Алгоритмы %%%

%\usepackage[linesnumbered]{algorithm2e}
\usepackage[linesnumbered,vlined,figure,scleft]{algorithm2e}

%% Glogal params %%
%ruled, tworuled, algoruled --- put lines to wrap the caption plus a line at the bottom (top) - one should not use this together with inline captions!
%vlined 										--- instead of begin...end will be lines
%boxed 											--- make a box
% figure 										--- count as Fig. ...


% Settings of caption       --- if one will use \caption{} option 	instead of inline + environment caption.
%\SetAlgoCaptionSeparator{.}
%\SetAlgorithmName{Algorithm}{} %last arg is the title of listing table


% Settings for lines numbers
\SetNlSkip{0em}							% sets the value of the space between the line numbers and the text, by default 1em.
\SetNlSty{normalsize}{\hphantom{0}}{.}%defines how to print line numbers
%\hspace*{5mm} does not work 
\SetAlgoNlRelativeSize{-1}	% sets the relative size of line numbers. By default, line numbers are two size smaller than algorithm text

% How to ignore line nuber and to wrap
%http://tex.stackexchange.com/questions/153646/algorithm2e-disabling-line-numbers-for-specific-lines
%http://tex.stackexchange.com/questions/86580/how-to-adjust-line-numbers-of-algorithm2e-package
\makeatletter
%\newcommand{\nosemic}{\renewcommand{\@endalgocfline}{\relax}}% Drop semi-colon ;
%\newcommand{\dosemic}{\renewcommand{\@endalgocfline}{\algocf@endline}}% Reinstate semi-colon ;
%\newcommand{\pushline}{\Indp}% Indent
%\newcommand{\popline}{\Indm\dosemic}% Undent
\let\oldnl\nl% Store \nl in \oldnl
\newcommand{\nonl}{\renewcommand{\nl}{\let\nl\oldnl}}% Remove line number for one line
\makeatother


% Settings for vlines 			
%\SetInd{0.3em}{0.5em}			%default and spaces before and after are 0.5em and 1em
%\SetVlineSkip{5em}					% Sets the value of the vertical space after the little horizontal line which closes a block in vlined mode

%% User abbreviations for ASTRA %%
\SetKwInput{KwInput}{Input}
\SetKwInput{KwOutput}{Output}
%% See also %%
%http://tex.stackexchange.com/questions/145114/caption-below-boxed-algorithm2e-when-used-as-a-figure
%http://tex.stackexchange.com/questions/83536/align-comments-in-algorithm-with-package-algorithm2e



