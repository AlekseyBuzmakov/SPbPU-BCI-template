%%% Прикладные пакеты %%% 
%\usepackage{calc}               % Пакет для расчётов параметров, например длины

%%% Для добавления Стр. над номерами страниц в оглавлении
%%% http://tex.stackexchange.com/a/306950
\usepackage{afterpage}

\urlstyle{rm} % links in Times


%\makeatletter
%%расстояние после ToC title до 1ой строчки 
%%для достижения одинаковых отсупов переопределено формирование базового ToC
%\renewcommand{\aftertoctitle}{\par\nobreak\vskip1\curtextsize}
%\makeatother

%https://tex.stackexchange.com/questions/170912/contents-page-in-two-different-languages
%\makeatletter
\newcommand\russiantableofcontents{%
%	\if@twocolumn
%	\@restonecoltrue\onecolumn
%	\else
%	\@restonecolfalse
%	\fi
	%  \begin{otherlanguage}{russian}
	\chapter*{%
	\normalfont\MakeUppercase{Содержание} %слово <<Содержание>> в стилю chaperLight, по факту убираем \bfseries
%		    \contentsname
%		    \@mkboth{\MakeUppercaseСодержание}
%		            {\MakeUppercaseСодержание}%
	}%
%\hrule
\vspace*{-1\curtextsize} %убрать лишний отступ в таблице
	\@starttoc{tuc}%
	%  \end{otherlanguage}
%	\if@restonecol\twocolumn\fi
}
\newcommand{\addtocru}[2]{%
	\addcontentsline{tuc}{#1}{\protect\numberline{\csname the#1\endcsname}#2}%
%	\addcontentsline{tuc}{#1}{#2}%
}
\newcommand{\addtocruNoProtect}[2]{%
%	\addcontentsline{tuc}{#1}{\protect\numberline{\csname the#1\endcsname}#2}%
		\addcontentsline{tuc}{#1}{#2}%
}

%обеспечение красивого порядка вывода содержаний и названий разделов, подразделов и т.п.
\newcommand\englishtableofcontents{%
	%	\if@twocolumn
	%	\@restonecoltrue\onecolumn
	%	\else
	%	\@restonecolfalse
	%	\fi
	%  \begin{otherlanguage}{russian}
	\chapter*{%
		\normalfont\MakeUppercase{Content} %слово <<Содержание>> в стилю chaperLight, по факту убираем \bfseries
		%		    \contentsname
		%		    \@mkboth{\MakeUppercaseСодержание}
		%		            {\MakeUppercaseСодержание}%
	}%
	%\hrule
	\vspace*{-1\curtextsize} %убрать лишний отступ в таблице
	\@starttoc{tec}%
	%  \end{otherlanguage}
	%	\if@restonecol\twocolumn\fi
}
\newcommand{\addtocen}[2]{%
		\addcontentsline{tec}{#1}{\protect\numberline{\csname the#1\endcsname}#2}%
%	\addcontentsline{tec}{#1}{#2}%
}
\newcommand{\addtocenNoProtect}[2]{%for preface, introduction etc
%	\addcontentsline{tec}{#1}{\protect\numberline{\csname the#1\endcsname}#2}%
		\addcontentsline{tec}{#1}{#2}%
}


%стандартный вывод в toc можно использовать, если издание только на английском или русском.
%переопределена, чтобы обеспечить одинаковые отсупы от названия ToC (toc, tec, tuc) до первой строки
\renewcommand\tableofcontents{%
	%	\if@twocolumn
	%	\@restonecoltrue\onecolumn
	%	\else
	%	\@restonecolfalse
	%	\fi
	%  \begin{otherlanguage}{russian}
	\chapter*{%
		\normalfont\MakeUppercase{Content} %слово <<Содержание>> в стилю chaperLight, по факту убираем \bfseries
		%		    \contentsname
		%		    \@mkboth{\MakeUppercaseСодержание}
		%		            {\MakeUppercaseСодержание}%
	}%
	%\hrule
	\vspace*{-1\curtextsize} %убрать лишний отступ в таблице
	\@starttoc{toc}%
	%  \end{otherlanguage}
	%	\if@restonecol\twocolumn\fi
}
%\newcommand{\addetoc}[2]{%
%	%	\addcontentsline{toc}{#1}{\protoct\numberline{\csname the#1\endcsname}#2}%
%	\addcontentsline{toc}{#1}{#2}%
%}


%\makeatother

%http://latex.org/forum/viewtopic.php?t=5438